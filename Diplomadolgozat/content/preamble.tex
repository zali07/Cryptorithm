%--------------------------------------------------------------------------------------
% Page layout setup
%--------------------------------------------------------------------------------------
% we need to redefine the pagestyle plain
% another possibility is to use the body of this command without \fancypagestyle
% and use \pagestyle{fancy} but in that case the special pages
% (like the ToC, the References, and the Chapter pages)remain in plane style

\usepackage{smartdiagram}
\usepackage{tikz,pgf}
\usepackage{pgfplots}
\pgfplotsset{width=7cm,compat=1.8}
\usetikzlibrary{matrix,calc,shapes}

\tikzset{
	treenode/.style = {shape=rectangle, rounded corners, draw, anchor=center, text width=5em, align=center, top color=white, bottom color=blue!20,inner sep=1ex},
	decision/.style = {treenode, diamond, inner sep=0pt},
	root/.style = {treenode, font=\Large, bottom color=red!30},
	env/.style = {treenode, font=\ttfamily\normalsize},
	finish/.style = {root, bottom color=green!40},
	dummy/.style = {circle,draw}
}


\setcounter{secnumdepth}{0}
\sectionfont{\large\upshape\bfseries}
\setcounter{secnumdepth}{2}

\sloppy % Margón túllógó sorok tiltása.
\widowpenalty=10000 \clubpenalty=10000 %A fattyú- és árvasorok elkerülése
\def\hyph{-\penalty0\hskip0pt\relax} % Kötőjeles szavak elválasztásának engedélyezése


%--------------------------------------------------------------------------------------
% Setup hyperref package
%--------------------------------------------------------------------------------------
\usepackage{xcolor}
\definecolor{bluecite}{HTML}{0875b7}
\usepackage[unicode=true,
bookmarksopen={true},
pdffitwindow=true, 
colorlinks=true, 
linkcolor=bluecite, 
citecolor=bluecite, 
urlcolor=bluecite, 
hyperfootnotes=false, 
pdfstartview={FitH},
pdfpagemode= UseNone]{hyperref}


%--------------------------------------------------------------------------------------
% Set up listings
%--------------------------------------------------------------------------------------



\definecolor{codegreen}{rgb}{0,0.6,0}
\definecolor{codegray}{rgb}{0.5,0.5,0.5}
\definecolor{codepurple}{rgb}{0.58,0,0.82}
\definecolor{backcolour}{rgb}{0.95,0.95,0.92}




\definecolor{lightgray}{rgb}{0.95,0.95,0.95}
\definecolor{darkgreen}{RGB}{3,125,80}
\lstset{frame=tb,
	language=Matlab,
	aboveskip=3mm,
	belowskip=3mm,
	showstringspaces=false,
	columns=flexible,
	basicstyle={\small\ttfamily},
	numbers=none,
	numberstyle=\tiny\color{gray},
	keywordstyle=\color{blue},
	commentstyle=\color{codegreen},
	%stringstyle=\color{mauve},
	breaklines=true,
	breakatwhitespace=true,
	tabsize=3,
	backgroundcolor=\color{lightgray},
}
\def\lstlistingname{k\'odr\'eszlet}	


%--------------------------------------------------------------------------------------
% Set up theorem-like environments
%--------------------------------------------------------------------------------------
% Using ntheorem package -- see http://www.math.washington.edu/tex-archive/macros/latex/contrib/ntheorem/ntheorem.pdf
%\swapnumbers
%\theoremstyle{plain}
%\theoremseparator{.}
\newtheorem{example}{\pelda}[section]

%\theoremseparator{.}
%\theoremprework{\bigskip\hrule\medskip}
%\theorempostwork{\hrule\bigskip}
%\theorembodyfont{\upshape}
%\theoremsymbol{{\large \ensuremath{\centerdot}}}
\newtheorem{definition}{\definicio}[section]

%\theoremseparator{.}
%\theoremprework{\bigskip\hrule\medskip}
%\theorempostwork{\hrule\bigskip}
\newtheorem{theorem}{\tetel}[section]

\newtheorem{conclusion}{Következtetés}[section]


%--------------------------------------------------------------------------------------
% Some new commands and declarations
%--------------------------------------------------------------------------------------
\newcommand{\code}[1]{{\upshape\ttfamily\scriptsize\indent #1}}
\newcommand{\doi}[1]{DOI: \href{http://dx.doi.org/\detokenize{#1}}{\raggedright{\texttt{\detokenize{#1}}}}} % A hivatkozások közt így könnyebb DOI-t megadni.

\DeclareMathOperator*{\argmax}{arg\,max}
%\DeclareMathOperator*[1]{\floor}{arg\,max}
\DeclareMathOperator{\sign}{sgn}
\DeclareMathOperator{\rot}{rot}


%--------------------------------------------------------------------------------------
% Setup captions
%--------------------------------------------------------------------------------------

\captionsetup[figure]{
	width=.75\textwidth,
	aboveskip=10pt}
\renewcommand{\captionlabelfont}{\bf}
%\renewcommand{\captionfont}{\footnotesize\it}


%--------------------------------------------------------------------------------------
% Redefine reference style
%--------------------------------------------------------------------------------------
\newcommand{\figref}[1]{\ref{fig:#1}.}
\renewcommand{\eqref}[1]{(\ref{eq:#1})}
\newcommand{\listref}[1]{\ref{listing:#1}.}
\newcommand{\sectref}[1]{\ref{sect:#1}}
\newcommand{\tabref}[1]{\ref{tab:#1}.}
