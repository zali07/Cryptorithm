%----------------------------------------------------------------------------
\chapter{Szoftver}
%----------------------------------------------------------------------------

\section{Felhasználói követelmények}

\textbf{Regisztráció és bejelentkezés:} A felhasználók lehetőséget kapnak a regisztrációra az alkalmazásban, amely után bejelentkezhetnek a személyes fiókjukba. Ez lehetővé teszi számukra a mentett adatokhoz való hozzáférést.

\textbf{Kriptográfiai rendszerek kiválasztása:} A felhasználóknak elérhetőek különböző kriptográfiai rendszerek. Ehhez megjelenik egy választási menüt, ahol kényelmesen navigálhatnak a rendszerek között.

\textbf{Rendszerek kipróbálása és tesztelése:} A felhasználóknak lehetősége van arra, hogy kipróbálják és teszteljék a választott kriptográfiai rendszereket. Ehhez felhasználóbarát és interaktív felületet van biztosítva, ahol megadhatják a bemeneti adatokat, és láthatják a kimeneti eredményeket.

\textbf{Tanulási anyagok és információk elérhetősége:} Az alkalmazásnak tartalmaznia egy tanulói felületet, ahol a felhasználók elérhetik a kriptográfiai rendszerekhez kapcsolódó részletes információkat és példafeladatokat.

\textbf{Nyelvi támogatás:} Az alkalmazás lehetőséget ad a felhasználóknak arra, hogy különböző nyelveken használják az alkalmazást. Ehhez többnyelvű interfészt van biztosítva, amely lehetővé teszi a felhasználóknak a nyelv kiválasztását.

\section{Rendszerkövetelmény}
A Cryptorithm rendszerkövetelménye funkcionális és nem funkcionális részekre oszlik. A funkcionális része tartalmazza az alkalmazás fő céljait és funkcionalitását, míg a nem funkcionális rész kitér a rendszerrel szemben támasztott követelményekre, mint például a felhasználói élmény, többnyelvűség és a rendszer architektúrája.

\subsection{Funkcionális}
\textbf{Kriptográfiai eszközök:} Az alkalmazás választékot kínál a népszerű és ismert eszközökből az adatok titkosításához, visszafejtéséhez és transzformációjához. A felhasználók szabadon válogathatnak és kipróbálhatnak különböző lehetőségeket.

\textbf{Tanulói felület:} Az alkalmazás rendelkezik egy tanulói felülettel, ahol a felhasználók elmélyülhetnek a kriptográfiai rendszerek megértésében. Itt információkat olvashatnak és gyakorlati példákon keresztül sajátíthatják el ezeket a rendszereket.

\textbf{Többnyelvűség:} Az alkalmazás támogatja a többnyelvűséget, így elérhetővé válik idegen nyelvű felhasználók számára is. A felhasználók kiválaszthatják az anyanyelvüket az alkalmazás használatához.

\subsection{Nem funkcionális}

\textbf{Felhasználói élmény:} Az alkalmazás könnyen használható felhasználói felülettel rendelkezik. Az intuitív navigáció és a felhasználóbarát tervezés lehetővé teszi a felhasználók számára a könnyű kezelést és a zökkenőmentes interakciót az alkalmazással.

\textbf{Teljesítmény:} Az alkalmazás gyors és hatékony működése.

\textbf{Biztonság:} Felhasználói adatok védelme, jelszavak titkosítása.

\textbf{Skálázhatóság:} Az alkalmazás képes kezelni a megnövekedett felhasználói forgalmat és rugalmasan skálázódni.

\textbf{Hibatűrés:} Az alkalmazás ellenálló képessége a hibákhoz, hibakezelés és hibajavítás mechanizmusainak megléte.

\textbf{Dokumentáció:} Részletes felhasználói dokumentáció az alkalmazás használatáról, beállításokról és funkcionalitásról.


\section{A rendszer architektúrája}
Az alkalmazás architektúrája ezeket a komponenseket kombinálja, hogy a felhasználók kényelmesen használhassák a Cryptorithm alkalmazást. A felhasználói interfész réteg lehetővé teszi a felhasználók interakcióját az alkalmazással, az üzleti logika réteg hajtja végre a szükséges műveleteket és feldolgozza a kéréseket, míg az adatbázis réteg biztosítja az adatok tartós tárolását és kezelését. Az architektúra segít a rendszer komponenseinek szétválasztásában és a fejlesztés hatékonyságának növelésében.

\textbf{Felhasználói interfész:} Ez a rész felelős az alkalmazás felhasználói felületének megjelenítéséért és a felhasználóval való interakcióért. Az UI a webes alkalmazásban HTML, CSS és JavaScript segítségével van megvalósítva. Ez a réteg jeleníti meg a kriptográfiai rendszerek kiválasztására, tesztelésére és a tanulói anyagokhoz való hozzáférésre szolgáló felületeket.

\textbf{Üzleti logika:} Az üzleti logika réteg felelős az alkalmazás működéséért, a felhasználók által végrehajtott műveletek feldolgozásáért és az eredmények előállításáért. Itt találhatóak a szerveroldali Python fájlok, amelyek a kriptográfiai műveletek, adatfeldolgozás és adatbázis-interakciók végrehajtásáért felelősek. Flask keretrendszert használva ez a réteg kezeli az HTTP kéréseket és válaszokat, valamint a felhasználói műveletek feldolgozását.

\textbf{Adatbázis:} Az alkalmazás adatbázisában tárolódnak a felhasználói adatok, például a regisztrált felhasználók adatai, kriptográfiai rendszerek, felhasználói eredmények stb. Az adatbázis kezelésére az SQLAlchemy Python könyvtárat használjuk, amely lehetővé teszi a könnyű adatbázis-műveletek végrehajtását és az adatmodell definiálását.

