\pagenumbering{gobble}

\selectlanguage{magyar}
\hungarianParagraph

%----------------------------------------------------------------------------
% Abstract in Hungarian
%----------------------------------------------------------------------------

\chapter*{Kivonat}

A Cryptorithm egy online platform, melynek célja, hogy a felhasználók kriptográfiai rendszereket próbálhassanak ki, és egy kiterjedt tudásbázist kínál, hogy szabadon tanulhassanak azok, eredetéről, működéséről és felhasználásáról.

A szoftver választékot kínál a népszerű és ismert eszközökből az adatok titkosításához, visszafejtéséhez és transzformációjához, lehetővé téve a felhasználóknak, hogy szabadon válogassanak és kipróbáljanak különböző lehetőségeket.

A szoftver emellett rendelkezik egy tanulói felülettel, amely lehetővé teszi a felhasználók számára a kriptográfiai rendszerek mélyebb megértését. Itt információkat olvashatnak, és gyakorlati példákon keresztül sajátíthatják el ezeket a rendszereket. 

A fejlesztés során kiemelt figyelmet fordítottam a felhasználói élmény biztosítására, így a szoftver könnyen használható felhasználói felülettel rendelkezik. Emellett fontos szempont volt az, hogy az oldal támogassa a többnyelvűséget is, így az oldal elérhetővé válik idegen nyelű felhasználók számára is.

A diplomadolgozatom további részében részletesen bemutatom a szoftver architektúráját, a megvalósított funkciókat, valamint a szoftverhez tartozó tudnivalókat. Emellett a felhasználói dokumentáció is részét képezi a dolgozatnak.


\vfill
\selectlanguage{romanian}

%----------------------------------------------------------------------------
% Abstract in Romanian
%----------------------------------------------------------------------------
\chapter*{Rezumat}

Cryptorithm este o platformă online concepută pentru a permite utilizatorilor să încerce sistemele criptografice și oferă o bază de cunoștințe extinsă pentru a învăța în mod liber despre originea, funcționarea și utilizarea acestora.

Software-ul oferă o selecție de instrumente populare și bine cunoscute pentru criptarea, decriptarea și transformarea datelor, permițând utilizatorilor să aleagă și să încerce în mod liber diferite opțiuni.

De asemenea, software-ul dispune de o interfață de învățare care le permite utilizatorilor să dobândească o înțelegere mai profundă a sistemelor criptografice. Aici ei pot citi informații și pot învăța despre aceste sisteme prin exemple practice. 

În timpul dezvoltării, s-a acordat o atenție deosebită asigurării unei experiențe ușor de utilizat, astfel încât software-ul are o interfață ușor de utilizat. În plus, a fost important ca site-ul să suporte multilingvismul, astfel încât să fie accesibil utilizatorilor de limbi străine.

În restul tezei mele voi descrie în detaliu arhitectura software-ului, caracteristicile implementate și cunoștințele despre software. În plus, documentația utilizatorului va face parte din teză.

\vfill
\selectlanguage{english}
%\englishParagraph

%----------------------------------------------------------------------------
% Abstract in English
%----------------------------------------------------------------------------
\chapter*{Abstract}

Cryptorithm is an online platform designed to allow users to try out cryptographic systems and offers an extensive knowledge base to freely learn about their origin, operation and use.

The software offers a selection of popular and well-known tools for encrypting, decrypting and transforming data, allowing users to freely choose and try out different options.

The software also features a learning interface that allows users to gain a deeper understanding of cryptographic systems. Here they can read information and learn about these systems through practical examples. 

During development, particular attention was paid to ensuring a user-friendly experience, so the software has an easy-to-use interface. In addition, it was important that the site supports multilingualism, so that it is accessible to foreign language users.

In the rest of my thesis I will describe in detail the architecture of the software, the implemented features and the knowledge about the software. In addition, the user documentation will be part of the thesis.

\vfill
\dolgozatnyelve
\defaultParagraph