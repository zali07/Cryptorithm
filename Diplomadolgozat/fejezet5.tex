%----------------------------------------------------------------------------
\chapter{Tervezés és megvalósítás}
%----------------------------------------------------------------------------
\section {Könyvtárak}

\begin{itemize}
  	\item\textbf{Cryptography}
A Cryptography egy Python könyvtár, amely kriptográfiai funkciókat és algoritmusokat kínál. Ez a könyvtár ideális választás a Cryptorithm projektben, mivel számos kriptográfiai műveletet valósíthatunk meg vele, például hashelést, titkosítást és visszafejtést. A Cryptography megbízható és jól dokumentált eszköz a kriptográfiai műveletek biztonságos végrehajtásához.

 	 \item\textbf{Flask}
A Flask egy könnyű súlyú, de erőteljes webes alkalmazások fejlesztésére szolgáló Python mikrokeretrendszer. A projektben a Flask keretrendszert használjuk a webes alkalmazás felépítéséhez és a kérések kezeléséhez. A Flask könnyen tanulható és rendkívül rugalmas, ami lehetővé teszi a fejlesztők számára, hogy könnyedén kialakítsák a vágyott funkciókat és testreszabhassák az alkalmazást.

 	 \item\textbf{FlaskSQLAlchemy}
A FlaskSQLAlchemy egy könnyen használható és hatékony ORM (Object-Relational Mapping) könyvtár, amely lehetővé teszi az adatbázis műveletek kezelését a Flask alkalmazásban. Az SQLAlchemy révén a FlaskSQLAlchemy segít az adatbázis kapcsolatok kezelésében, az adatmodell osztályok definiálásában és az adatbázis műveletek végrehajtásában. A FlaskSQLAlchemy használata átlátható és hatékony adatbázis-interakciókat tesz lehetővé a projektben.

 	 \item\textbf{FlaskLogin}
A FlaskLogin egy hasznos kiegészítő a Flask keretrendszerhez, amely segít az autentikáció és az azonosítás kezelésében a webes alkalmazásban. A FlaskLogin segítségével egyszerűen implementálhatunk felhasználói regisztrációt, bejelentkezést és kijelentkezést a rendszerben. Ez a könyvtár nagyban megkönnyíti a felhasználói munkamenetek kezelését és a hozzáférési jogosultságok ellenőrzését.

 	 \item\textbf{Json}
A JSON (JavaScript Object Notation) egy könnyen olvasható és írható adatformátum, amely széles körben használatos az adatok strukturált tárolására és átvitelére. A JSON könyvtárat használva a projektben könnyedén kezelhetjük a JSON adatokat, például menthetjük és betölthetjük az alkalmazás beállításait vagy az egyéb adatokat.

 	 \item\textbf{Base64}
A base64 egy olyan kódolási formátum, amely lehetővé teszi bináris adatok átalakítását olvasható szöveggé. A base64 könyvtárat használva a projektben a bináris adatokat (például képek vagy fájlok) konvertálhatjuk base64 formátumba, ami könnyen kezelhető és továbbítható a webes alkalmazásban.

 	 \item\textbf{Glob}
A glob egy Python könyvtár, amely lehetővé teszi a fájl- és mappaútvonalak kezelését. A glob könyvtár segítségével könnyedén kezelhetjük a fájlok vagy mappák neveinek gyűjteményét, és feldolgozhatjuk azokat a projektben szükséges műveletek végrehajtásához.

 	 \item\textbf{Os}
Az os (operációs rendszer) egy Python könyvtár, amely lehetővé teszi az operációs rendszerrel kapcsolatos műveletek végrehajtását. Az os könyvtárat használva a projektben könnyedén kezelhetjük a fájlok és könyvtárak manipulációját, például fájlok létrehozását, törlését vagy módosítását.


 	 \item\textbf{Math}
A math egy beépített Python könyvtár, amely matematikai függvényeket és konstansokat kínál. A math könyvtárat használva a projektben matematikai műveleteket végezhetünk, például számításokat végezhetünk, véletlenszámokat generálhatunk vagy trigonometriai műveleteket hajthatunk végre.

 	 \item\textbf{Hashlib}
A hashlib egy Python könyvtár, amely hash-függvényeket kínál. A hashlib segítségével a projektben könnyedén végezhetünk különböző hashelési műveleteket, például SHA vagy MD5 hashelést, amelyek a jelszavak biztonságos tárolásában vagy az adatok hitelességének ellenőrzésében hasznosak lehetnek.

 	 \item\textbf{Random}
A random egy beépített Python könyvtár, amely véletlenszámok generálására szolgál. A random könyvtárat használva a projektben véletlenszerű adatokat generálhatunk, például jelszavakat vagy egyedi azonosítókat.

 	 \item\textbf{String}
A string egy beépített Python könyvtár, amely karakterláncokkal kapcsolatos műveleteket kínál. A string könyvtárat használva a projektben könnyedén kezelhetjük a karakterláncokat, például módosíthatjuk vagy manipulálhatjuk őket a szükséges adatmanipulációkhoz.

\end{itemize}


\section {Bemutatás}
%kezdooldal - login/register
%crypt
%learning

\section {Diagrammok}
%use case
%sequential
