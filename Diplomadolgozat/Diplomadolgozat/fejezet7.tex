%----------------------------------------------------------------------------
\chapter{Továbbfejlesztési lehetőségek}
%----------------------------------------------------------------------------


\textbf{Bővített kriptográfiai rendszerek:} Az alkalmazásba további kriptográfiai rendszerek, mint például Salsa20, ElGamal vagy ECC (Elliptic Curve Cryptography) integrálása lehetőséget adna a felhasználóknak a különböző algoritmusok kipróbálására és tanulására.

\textbf{Interaktív gyakorló feladatok:} Interaktív gyakorló feladatok készítése, amelyek segítségével a felhasználók a tanulói felületen gyakorolhatják a különböző kriptográfiai rendszerek alkalmazását. Ez javítaná a gyakorlati tapasztalatok szerzését és a tanulás hatékonyságát.

\textbf{Felhasználói profilok és eredmények nyomon követése:} Egy felhasználói profil rendszer létrehozása, ahol a felhasználók modosíthatják adataikat és nyomon követhetik a teljesítményt, például az elért eredményeket vagy az elvégzett feladatokat. Ez segít a felhasználóknak a fejlődésük nyomon követésében és motivációjuk fokozásában.

\textbf{További nyelvi támogatás:} Az alkalmazást további nyelvek támogatásával való kibővítése, hogy a nem angol vagy magyar anyanyelvű felhasználók is könnyen használhassák és megértsék a rendszert. Ez növelné a felhasználói kört és a felhasználói elégedettséget.

\textbf{Reszponzív dizájn:} Az alkalamzás biztosítása a különböző eszközökön történő használathoz.

\vspace{10pt}
Ezek a továbbfejlesztési lehetőségek hozzájárulhatnak az alkalmazás funkcionalitásának és felhasználói élményének javításához, valamint a felhasználók kriptográfiai ismereteinek bővítéséhez.