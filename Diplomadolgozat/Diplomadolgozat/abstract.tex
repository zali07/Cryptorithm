\pagenumbering{gobble}

\selectlanguage{magyar}
\hungarianParagraph

%----------------------------------------------------------------------------
% Abstract in Hungarian
%----------------------------------------------------------------------------

\chapter*{Kivonat}

A Cryptorithm egy online platform, ahol a felhasználók kriptográfiai rendszereket próbálhatnak ki, és egy kiterjedt tudásbázist kínál, hogy szabadon tanulhassanak azok, eredetéről, működéséről és felhasználásáról.

A szoftver választékot kínál a népszerű és ismert eszközökből az adatok titkosításához, visszafejtéséhez és transzformációjához, lehetővé téve a felhasználóknak, hogy szabadon válogathassanak és kipróbáljanak különböző lehetőségeket. Emellett rendelkezik egy tanulói felülettel, amely hozzájárul a felhasználók számára a kriptográfiai rendszerek mélyebb megértésében. Itt információkat olvashatnak, és gyakorlati példákon keresztül sajátíthatják el ezeket a rendszereket. 

Kiemelt figyelmet fordítottam a fejlesztés során a felhasználói élmény biztosítására, így a szoftver könnyedén használható felhasználói felülettel rendelkezik. Emellett fontos szempont volt az, hogy az oldal támogassa a többnyelvűséget is, így az oldal elérhetővé válik magyar és angol nyelű felhasználók számára is.

A diplomadolgozatom további részében részletesen bemutatom a szoftver architektúráját, a megvalósított funkciókat, valamint a hozzá tartozó tudnivalókat. Emellett a felhasználói dokumentáció is részét képezi a dolgozatnak.


\vfill
\selectlanguage{romanian}

%----------------------------------------------------------------------------
% Abstract in Romanian
%----------------------------------------------------------------------------
\chapter*{Rezumat}

Cryptorithm este o platformă online unde utilizatorii pot încerca sisteme criptografice și oferă o bază de cunoștințe extinsă pentru a învăța în mod liber despre originea, funcționarea și utilizarea acestora.

Software-ul oferă o selecție de instrumente populare și bine cunoscute pentru criptarea, decriptarea și transformarea datelor, permițând utilizatorilor să aleagă și să încerce în mod liber diferite opțiuni. De asemenea, dispune de o interfață de învățare pentru a-i ajuta pe utilizatori să dobândească o înțelegere mai profundă a sistemelor criptografice. Aici aceștia pot citi informații și pot învăța despre aceste sisteme prin exemple practice. 

Am acordat o atenție deosebită asigurării unei experiențe de utilizare în timpul dezvoltării, astfel încât software-ul are o interfață ușor de utilizat. În plus, a fost important ca site-ul să suporte multilingvismul, astfel încât să fie accesibil utilizatorilor de limbi străine.

În restul tezei mele voi descrie în detaliu arhitectura software-ului, caracteristicile implementate și cunoștințele aferente. În plus, documentația utilizatorului va face parte din teză.

\vfill
\selectlanguage{english}
%\englishParagraph

%----------------------------------------------------------------------------
% Abstract in English
%----------------------------------------------------------------------------
\chapter*{Abstract}

Cryptorithm is an online platform where users can try out cryptographic systems and offer an extensive knowledge base to freely learn about their origin, operation and use.

The software offers a selection of popular and well-known tools for encrypting, decrypting and transforming data, allowing users to freely choose and try out different options. It also has a learning interface to help users gain a deeper understanding of cryptographic systems. Here they can read information and learn about these systems through practical examples. 

I have paid particular attention to ensuring a user experience during development, so the software has an easy-to-use interface. In addition, it was important that the site supports multilingualism so that it is accessible to foreign language users.

In the rest of my thesis I will describe in detail the architecture of the software, the implemented features and the related knowledge. In addition, the user documentation will be part of the thesis.

\vfill
\dolgozatnyelve
\defaultParagraph