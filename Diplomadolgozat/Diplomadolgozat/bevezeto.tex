%----------------------------------------------------------------------------
\chapter{Bevezető}%\addcontentsline{toc}{chapter}{Bevezető}
%----------------------------------------------------------------------------

A digitális korban a kriptográfia egyre növekvő jelentőséggel bír, hiszen az adatbiztonság és az adatvédelem kulcsfontosságú tényezők az információtechnológia területén. A kriptográfiai rendszerek megértése és hatékony használata elengedhetetlen az adatok titkosításához, visszafejtéséhez és biztonságos átvételéhez.

Célja a dolgozatomnak, hogy bemutassa a Cryptorithm nevű webes alkalmazást, amely egy kiterjedt platformot kínál a kriptográfiai rendszerek gyakorlati kipróbálására és tanulására. Az alkalmazás lehetővé teszi a felhasználók számára, hogy interaktív módon felfedezzék a különböző kriptográfiai algoritmusokat és azok alkalmazási területeit, mint például az SHA, Whirlpool és bcrypt hash fügvények, a Caesar, Affin és ChaCha20 folyamtitkosítók vagy az AES és Blowfish blokktitkosítók. Emellett a Cryptorithm egy részletes tudásbázist is nyújt, amely segíti a felhasználókat az algoritmusok eredetének, működésének és felhasználásának megértésében.

A Cryptorithm alkalmazás a Flask, egy Python alapú könnyű súlyú webes keretrendszerre épül, amely lehetővé teszi a gyors és hatékony fejlesztést. Az alkalmazás intuitív és felhasználóbarát felülettel rendelkezik, hogy könnyű legyen a navigáció és a kriptográfiai rendszerek gyakorlati alkalmazása. Az alkalmazás többnyelvű támogatást is biztosít, így bárki, függetlenül az anyanyelvétől, elérheti és használhatja a platformot.

\begin{figure}[!h]
	\centering
	\includegraphics[scale=0.25]{images/logoCryptorithm}
	\caption{Cryptorithm logó}
\end{figure}
\pagebreak

%----------------------------------------------------------------------------
\section {Motiváció és Célkitűzések}
%----------------------------------------------------------------------------
\textbf{Kriptográfiai tudatosság növelése:}
Az alkalmazás célja, hogy növelje a felhasználók kriptográfiai tudatosságát és ismereteit. A projekt fő motivációja a felhasználók tájékozottságának növelése a digitális biztonság terén, segítve őket a kriptográfiai rendszerek megértésében és alkalmazásában.

\textbf{Gyakorlati tapasztalat nyújtása:}
Lehetőséget kapnak a felhasználók, hogy gyakorlatban is kipróbálják és megtapasztalják a kriptográfiai rendszerek működését.

\textbf{Tanulás és oktatás támogatása:}
Egy tanulói felületet van biztosítva, ahol a felhasználók elmélyülhetnek a kriptográfia terén. Motivációs tényező a tudásátadás és az oktatás támogatása, amely segít a felhasználóknak a kriptográfiai rendszerek megértésében és elsajátításában.

\textbf{Nyitottság és közösség támogatása:}
Az alkalmazás lehetőséget nyújt az idegen anyanyelvű felhasználóknak is a használatra, hogy bővítsék tudásukat a kriptográfiai rendszerekről.






